  CMake is an open source build manager for software projects that allows developers to specify build parameters in a simple protable text file format. It can handle several difficult aspects of building software such as 
  \begin{itemize}  
    \item cross platform builds
    \item system introspection
    \item user customized builds  
  \end{itemize}  
  according to a user-friendly script file. \par
  It's a unified build system, which helps to eliminate the need of maintaining platform-specifc build systems, such as, the \texttt{Makefile} for UNIX and \texttt{workspace} for Microsoft Visual Studio.\par
  CMake provides many benefits for single platform multi-machine development environments including:
	\begin{itemize}
    \item Search automatically for programs, libraries, and header files that may be required by the software being built. And also onsider environment variables and Window's registry settings when searching.
		\item Enable building in a directory tree outside of the source tree. 
		\item Extended by complex custom commands for automatically generated new source files during the build process and then are compiled into the software.
		\item Allow users to select optional components at configuration time. 
		\item Generate workspaces and projects automatically from a simple text file. 
		\item Switch easily between static and shared builds. 
		\item Build up file dependencies automatically and support for parallel builds on most platforms.
	\end{itemize}
	When developing cross platform software, CMake provides a number of additional features:
  \begin{itemize}
		\item The ability to check for hardware specific characteristics like machine byte order.
		\item A single set of build configuration files that work on all platforms. 
		\item Support for building shared libraries on all platforms that support it.
    \item The ability to configure files with system dependent information, such as the location of data files and other information (e.g., macro definitions by \mkeyword{\#define} in C++). 
	\end{itemize}
\section{The history of CMake}
  CMake development began in 1999 as part of the Insight Toolkit (\href{https://itk.org/}{ITK}) funded by the US National Library of Medicine. 
\section{Why Not Others}
\subsection{Autoconf}	
	Autoconf combined with automake provides some of the same fimctionality as CMake, but 
  \begin{itemize}  
    \item Requires the installation of many additional tools not found natively on a Windows box. 
    \item Difficult to use or extend and impossible for some tasks that are easy in CMake. 
    \item Generates \texttt{Makefiles} that will force users to the command line.
    \item Does not support dependent options where one option depends on some other property or selection.
  \end{itemize}  
  More for UNIX users, CMake provides
  \begin{itemize}  
    \item Automated dependency generation that is not done directly by autoconf.
    \item Simple input format is easier to read and maintain than a combination of \texttt{Makefile.in} and \texttt{configure.in} files. 
    \item The ability to remember and chain library dependency information has no equivalent in autoconf/automake.
  \end{itemize}  
\subsection{JAM, qmake, SCons, or ANT}
   Many of these tools require other tools such as Python or Java to be installed before they will work.
\subsection{Script It Yourself}
  \begin{itemize}  
    \item Dependencies management is better done by CMake.
    \item CMake would require no more tools than 
      \begin{itemize}  
        \item a C compiler, that compiler's native build tools
        \item a CMake executable. CMake was written in C++, requires only a C++ compiler to build and precompiled binaries are available for most systems.  
      \end{itemize}  
    \item Self Scripting typically means platform-dependence, limiting its application to building in Mac and Windows.
  \end{itemize}  
\section{Platforms Requirement}
  Most OSs, including
  \begin{itemize}  
    \item Microsoft Windows
    \item Apple Mac OS X
    \item Most UNIX or UNIX-like platforms
  \end{itemize}  
  And most common compilers, such as 
  \begin{itemize}  
    \item GNU compilers 
    \item Visual Studio
  \end{itemize}  

